
\documentclass[a4paper, 11pt]{article}
\usepackage[ascii]{inputenc}
\usepackage{supertabular}
\usepackage[ngerman]{babel}
\usepackage{amsmath}
\usepackage{amssymb,amsfonts,textcomp}
\usepackage {geometry}
\geometry{a4paper,top=25mm,left=30mm,right=25mm,bottom=30mm}
\usepackage{color}
\usepackage{array}
\usepackage{hhline}
\usepackage{hyperref}
\hypersetup{colorlinks=true, linkcolor=blue, citecolor=blue, filecolor=blue, urlcolor=blue}


\begin{document}
{\begin{center}\huge\bf openETCS Scrum Meeting Minutes\end{center}}
\section{Meeting Organisation}

\renewcommand{\arraystretch}{1.5}
\begin{supertabular}{m{.2\textwidth}m{.8\textwidth}}
%\hline
Subject & Weekly Scrum of Scrums\\
Organised by & Bernd Hekele\\
Minutes by & Bernd Hekele\\
When & August, 8th, 2014, 10:00 - 11:45\\
Location & Telco and Goto-Meeting\\
%9:30-10:00 and 
%\hline
\end{supertabular}

\renewcommand{\arraystretch}{1.0}
\section{Meeting:}

\begin{tabular}{|c|l|l|}
\hline
\textbf{Slot} &  \textbf{Called By} & \textbf{Workpackage} \\
\hline  
10:00 - 10:15 & Stefan Rieger & WP6: Dissemination \\\hline  
10:15 - 10:30 & Nicolas van Landeghem & WP5: Demonstration \\\hline  
10:30 - 10:45 & Marc Behrens & WP4: Verification \& Validation \\\hline  
10:45 - 11:00 & Fausto Cochetti & WP3: Modelling \\\hline  
%10:45 - 11:00 & Bernd Hekele & WP3: Modelling \\\hline  
11:00 - 11:15 & Michael Jastram & WP7: Toolchain \\\hline
%11:15 - 11:30 & Baseliyos Jacob & WP2: API Requirements \\\hline  
11:15 - 11:45 & Klaus-R\"udiger Hase & PMB: Project Management Board \\\hline  
%11:15 - 11:45 & Bernd Hekele & PMB: Project Management Board \\\hline  
\end{tabular}

\section{Participants in Meeting:}

\begin{tabular}{|l|c|c|c||c|c|c||c|c|c|}
\hline
\textbf{Participant} & \textbf{WP6} &  \textbf{WP5} & \textbf{WP4}&  \textbf{WP3} & \textbf{WP7}&  \textbf{PMB} \\\hline
% Name                  6   5   4   3   7   P
Alexander Nitsch     & x & x & x & x & x & x \\\hline  
Alexander Stante     &   &   &   &   & x &   \\\hline 
Baseliyos Jacob      &   &   &   & x & x & x \\\hline 
%Bernd Gonska         &   &   & x &   &   &   \\\hline
Bernd Hekele         & x &   & x & x & x & x \\\hline
Cecile Braunstein    &   &   & x & x & x &   \\\hline
%Christian Giraud     &   &   &   &   &   &   \\\hline
Christian Stahl      &   &   &   & x &   &   \\\hline
%David Mentre         &   &   &   &   &   &   \\\hline
Fausto Cochetti      &   &   &   & x & x & x \\\hline
Hardi Hungar         &   &   & x &   &   &   \\\hline
%Jos Holtzer         & x &   &   &   &   &   \\\hline
%Huu-Nghia Nguyen     & x &   & x & x & x &   \\\hline
%Izaskun de la Torre  &   &   &   & x & x &   \\\hline
Jan Welte            & x & x & x & x & x & x \\\hline
%Jan Welvaarts        & x & x & x & x &   &   \\\hline
Jens Gerlach         & x & x & x & x &   &   \\\hline
%Jonas Helming        &   &   &   &   &   &   \\\hline
%Klaus-R\"udiger Hase & x &   & x & x & x & x \\\hline
%Laurent Ferier       &   &   &   & x & x & x \\\hline
Lukasz Fronc         &   &   &   &   & x & x \\\hline
Marc Behrens         &   & x & x & x & x & x \\\hline
%Marielle Petit-Doche &   & x & x & x & x & x \\\hline
%Mathilde Arnaud      &   &   &   &   &   &   \\\hline
%Matthieu Perin       &   &   & x & x & x & x \\\hline
Michael Jastram      &   &   &   & x & x & x \\\hline
Nicolas van Landeghem& x & x & x & x & x & x \\\hline
%Peter Mahlmann       &   & x & x & x & x & x \\\hline
%Peyman Farhangi      &   &   & x & x & x & x \\\hline
Pierre-Yves Le Morvan&   & x & x & x & x & x \\\hline
%Silvano dal Zilio    &   &   & x & x & x & x \\\hline
Stefan Rieger        & x & x & x & x & x & x \\\hline
Uwe Steinke          & x & x & x & x & x & x \\\hline
%Veronique Gontier    & x & x & x & x & x & x \\\hline
%Vincent Nuhaan       &   &   &   & x &   &   \\\hline
% Name                  6   5   4   3   7   P  \\\hline
\end{tabular}

%\line(1,0){440}

\section{Scrum Meetings}

\subsection{WP6: Dissemination}

\begin{itemize}
\item openETCS@Innotrans: Speakers Corner\\
You find a regular update of the planning of this event in the dissemination Wiki:
\url{https://github.com/openETCS/Dissemination/wiki/INNOTRANS-2014-Planning}.

Jos Holtzer (NS) confirmed NS to step in to host the speaker's corner. Klaus-R\"udiger also will plan for a talk. Details are to be planned to be discussed in a special meeting in the afternoon. The meeting results are to be made available in the dissemination wiki. Clarification of rooms for the mid-term workshop is still ongoing.

\end{itemize}

\subsection{WP5: Demonstration}
\begin{itemize}
\item Update on Activities\\
Architecure work is progressing. ERSA will use the openETCS tool for documenting the architecture (SysML). Feedback is requested on the API (use issue \url{https://github.com/openETCS/demonstrator/issues/13})
Discussing first findings on the API the conclusion was to improve the principles and concept respectively  use cases.
\end{itemize}


\subsection{WP4: Validation \& Verification}
\begin{itemize}
\item actual activities\\
WP4 started to clear issues collected in the first sprint related to D4.2 and related to the internal assessment.

\end{itemize}

\subsection{WP3: Modelling}

\begin{itemize}
\item Activities\\
The API document v 1.3 is now on Github: \url{https://github.com/openETCS/requirements/tree/master/D2.7-Technical_Appendix}

For completing the description of the "initial architecture" a new document has ben started: \url{https://github.com/openETCS/modeling/tree/master/openETCS%20ArchitectureAndDesign/FirstIteration}.\\
Scope of the new document is the definition of constraints of the model, the documentation of design principles and the introduction to the models.

Site and date for the next WP3 modelling team meeting is fixed now:\\
rooms reserved in MAI = Maison des Associations International,\\
Rue Washington 40, 1050 IXELLES,\\
(Tram 81,94 Bus 54 stop BAILLI). The invitation and agenda draft will follow. SNCF plans to participate.

The document on database by Chrstian Giraud is available on Github (version 9).

\end{itemize}

\subsection{WP7: Toolchain}

\begin{itemize}
\item Activities\\
A new release of the tool-chain is available. 

For interfacing Scade models an plug-in is under test which generates the Scade models from SysML.  For releasing this function a clarification of licensing is needed.
Progress is updated in the sprint activity backlog.

In the meeting a demo on a compare functions inside openETCS toolchain was presented. The function is available from the shelf and shall be part of the openETCS tools-chain.

\end{itemize}

\subsection{WP2: Requirements}

%\begin{itemize}

No meeting this week.

%\end{itemize}

\subsection{PMB: Project Management Board}
\begin{itemize}
\item a.o.b.\\
The topic of Scade training was discussed. With a number of new partners (e.g., vie the openIT4SR project) additional need arises. Thew need for Scade Design training was discussed as well as the need for training focusing more on Scade verification. Projectoffice will collect requirements.

Ideas on the scrum-of-scrum concpet have been discussed further in a meeting (Marc and Bernd). For implementation a list of openETCS features being part of the backlog will be started - aligned with the view on priorities.
\end{itemize}

\section{Notes}
\begin{itemize}

\item Note: Location of Meeting Minutes\\
You can find these minutes here: \url{https://github.com/openETCS/governance/blob/master/scrumMeetings/Minutes/Minute_20140808.pdf}. Please, use the issue tracker for findings with the minutes.

\end{itemize}

\end{document}
